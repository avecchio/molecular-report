\documentclass[journal, a4paper]{IEEEtran}

%\usepackage{cite}
\usepackage{graphicx}
%\usepackage{psfrag}
%\usepackage{subfigure}
\usepackage{url}
%\usepackage{stfloats}
\usepackage{amsmath}
%\usepackage{array}

% Package Settings
\graphicspath{ {images/} }

\begin{document}

% Define document title and author
    \title{Testing the Effects of Transfection on Mammalian Cytokine RNA Expression}
    \author{Austin Mitchell Vecchio
    \thanks{Advisor: Dipl.--Ing.~Michelle Ammerman, Lehrstuhl f\"ur Nachrichtentechnik, TUM, WS 2050/2051.}}
    \markboth{Hauptseminar Digitale Kommunikationssysteme}{}
    \maketitle

% Write abstract here
\begin{abstract}
  This experiment aims to study the effect of genetic expression on polyplex treated cells against &\beta&-actin and INF&\alpha&.

\end{abstract}

% Each section begins with a \section{title} command
\section{Introduction}
    % \PARstart{}{} creates a tall first letter for this first paragraph
    %\PARstart{T}{his} section introduces the topic and leads the reader on to the main part.


For the immunological transcripts IL6 and TNFA will be used to see if they will have an effect on the amplification of DNA.


    The mRNA sampl

Test the effects on

Each student was given a sample as follows

Poly24 Av
Poly24 Ts
Just Cells
No Wash

% Main Part
\section{Methods}
    \subsection{Purification}
      mRNA

    \subsection{Reverse Transcription}
      The mRNA must be converted into cDNA.

    \subsection{Qt PCR}
    \begin{table}[!hbt]
      % Center the table
      \begin{center}
      \caption{cDNA Concentrations}
      \label{tab:simParameters}
      \begin{tabular}{|c|c|c|}
        \hline
        INFA & (Ts, AV, Mi) & 1.32 \\
        \hline
        INFA & (Ni) & 11.79 \\
        \hline
        $\beta$-Actin & (Ts, AV, Mi) & 2.00 \\
        \hline
        $\beta$-Actin & (Ni) & 2.9 \\
        \hline
      \end{tabular}
      \end{center}
    \end{table}

    The concentrations of cDNA for Ts, Av and Mi were all around 100ng/ $\mu$ L. Therefore one set of calculations can be used for all three
    experiments. The concentrations of Ni, however, were reported ot be 11.2ng/ $\mu$ L and thus, had to have a different set of calculations.

    \subsection{Analysis}

        Used $\Delta$\Delta$ct method was used in calculating fold inductions.

\section{Results}
  \subsection{Qubit}
    \begin{table}[!hbt]
      % Center the table
      \begin{center}
      % Fold Inductions
      \caption{Qubit Results}
      \label{tab:simParameters}
      % Table itself: here we have two columns which are centered and have lines to the left, right and in the middle: |c|c|
      \begin{tabular}{|c|c|}
      \hline
      SD1 & 54.27 \\
      \hline
      SD2 & 1057.52 \\
      \hline
      Tk & 8.75 \\
      \hline
      DA & 68.0 \\
      \hline
      AV & 487.0 \\
      \hline
      Tk & 28.0 \\
      \hline
      \end{tabular}
      \end{center}
    \end{table}

  \subsection{Q-PCR}
    \begin{table}[!hbt]
      % Center the table
      \begin{center}
      % Fold Inductions
      \caption{Qubit Results}
      \label{tab:simParameters}
      % Table itself: here we have two columns which are centered and have lines to the left, right and in the middle: |c|c|
      \begin{tabular}{|c|c|}
      \hline
      SD1 & 54.27 \\
      \hline
      SD2 & 1057.52 \\
      \hline
      Tk & 8.75 \\
      \hline
      DA & 68.0 \\
      \hline
      AV & 487.0 \\
      \hline
      Tk & 28.0 \\
      \hline
      \end{tabular}
      \end{center}
    \end{table}




\section{Discussion}

For a future experiment, it would be suggested that the polyplex cells are treated in 3 hour increments from 3 hours to 24 hours.


\section{Conclusion}
The mRNA was purified and converted to cDNA. THe resulting concentration was relatively high compared to peers. This eludes that the
treatment for these cells of polycationic DNA for 24 hours could result in higher transcription rates.


\section{Figures}
  \begin{table}[!hbt]
    % Center the table
    \begin{center}
    % Fold Inductions
    \caption{Simulation Parameters}
    \label{tab:simParameters}
    % Table itself: here we have two columns which are centered and have lines to the left, right and in the middle: |c|c|
    \begin{tabular}{|c|c|c|c|c|c|c|c|c|c|c|c|c|}
      \hline
      & 1 & 2 & 3 & 4 & 5 & 6\\
      \hline
      A & 4ng/$\mu$L & 4ng/$\mu$L & 4ng/$\mu$L & 4ng/$\mu$L & 4ng/$\mu$L & 4ng/$\mu$L\\
      \hline
      B & 4ng/$\mu$L & 4ng/$\mu$L & 4ng/$\mu$L & 4ng/$\mu$L & 4ng/$\mu$L & 4ng/$\mu$L\\
      \hline
      C & 4ng/$\mu$L & 4ng/$\mu$L & 4ng/$\mu$L & 4ng/$\mu$L & 4ng/$\mu$L & 4ng/$\mu$L\\
      \hline
      D & 4ng/$\mu$L & 4ng/$\mu$L & 4ng/$\mu$L & 4ng/$\mu$L & 4ng/$\mu$L & 4ng/$\mu$L\\
      \hline
      E & Ts & Ni & AV & Ts & Ni & AV\\
      \hline
      E & Ts & Ni & AV & Ts & Ni & AV\\
      \hline
      G & Mi -RT & Mi NTC & & Mi -RT & Mi NTC & \\
      \hline
      H & & & & & &\\
      \hline
    \end{tabular}
    \end{center}
  \end{table}

  \begin{table}[!hbt]
    % Center the table
    \begin{center}
    % Fold Inductions
    \caption{Simulation Parameters}
    \label{tab:simParameters}
    % Table itself: here we have two columns which are centered and have lines to the left, right and in the middle: |c|c|
    \begin{tabular}{|c|c|c|c|c|c|c|c|c|c|c|c|c|}
      \hline
      & 7 & 8 & 9 & 10 & 11 & 12 \\
      \hline
      A & 4ng/$\mu$L & 4ng/$\mu$L & 4ng/$\mu$L & 4ng/$\mu$L & 4ng/$\mu$L & 4ng/$\mu$L\\
      \hline
      B & 4ng/$\mu$L & 4ng/$\mu$L & 4ng/$\mu$L & 4ng/$\mu$L & 4ng/$\mu$L & 4ng/$\mu$L\\
      \hline
      C & 4ng/$\mu$L & 4ng/$\mu$L & 4ng/$\mu$L & 4ng/$\mu$L & 4ng/$\mu$L & 4ng/$\mu$L\\
      \hline
      D & 4ng/$\mu$L & 4ng/$\mu$L & 4ng/$\mu$L & 4ng/$\mu$L & 4ng/$\mu$L & 4ng/$\mu$L\\
      \hline
      E & Ts & Ni & AV & Ts & Ni & AV\\
      \hline
      E & Ts & Ni & AV & Ts & Ni & AV\\
      \hline
      G & Mi -RT & Mi NTC & & Mi -RT & Mi NTC & \\
      \hline
      H & & & & & &\\
      \hline
    \end{tabular}
    \end{center}
  \end{table}


% Now we need a bibliography:
\begin{thebibliography}{5}
    %Each item starts with a \bibitem{reference} command and the details thereafter.
    \bibitem{HOP96} % Transaction paper
    J.~Hagenauer, E.~Offer, and L.~Papke. Iterative decoding of binary block
    and convolutional codes. {\em IEEE Trans. Inform. Theory},
    vol.~42, no.~2, pp.~429–-445, Mar. 1996.

    \bibitem{MJH06} % Conference paper
    T.~Mayer, H.~Jenkac, and J.~Hagenauer. Turbo base-station cooperation for intercell interference cancellation. {\em IEEE Int. Conf. Commun. (ICC)}, Istanbul, Turkey, pp.~356--361, June 2006.

    \bibitem{Proakis} % Book
    J.~G.~Proakis. {\em Digital Communications}. McGraw-Hill Book Co.,
    New York, USA, 3rd edition, 1995.

    \bibitem{talk} % Web document
    F.~R.~Kschischang. Giving a talk: Guidelines for the Preparation and Presentation of Technical Seminars.
    \url{http://www.comm.toronto.edu/frank/guide/guide.pdf}.

    \bibitem{5}
    IEEE Transactions \LaTeX and Microsoft Word Style Files.
    \url{http://www.ieee.org/web/publications/authors/transjnl/index.html}

\end{thebibliography}

% Your document ends here!
\end{document}
